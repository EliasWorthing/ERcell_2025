
% ----- preamble -----
% {{{
\documentclass[journal=mamobx, layout=twocolumn]{achemso}

\usepackage{lipsum} % for generating dummy text
\usepackage[dvipsnames]{xcolor} % for colors
\usepackage{graphicx} % for figures
\usepackage{amsmath, amssymb} % for math
\usepackage{bm} % for bold math fonts
\usepackage{xr} % crossreferencing main manuscript to and from supporting info
\externaldocument[SI:]{suppinfo}
% }}}

% --- Title, Author Info, Abstract Text ---
% {{{
\title{On the Development of a Molten Salt Electrorefining Simulation}
\author{Dakota S. Banks}
\affiliation{Chemical Engineering Department, Brigham Young University, Provo, Utah}
\date{\today}
\email{dakotasb@byu.edu}

\newcommand*{\abstracttext}{
% Put abstract text here
This document provides instructions for making an \emph{outline} of a paper.
The instructions here correspond to Step $3$ of the process described in the README and in Whitesides article.
Once you have an outline (and data of course), writing a paper is a relatively straightforward (but often time-consuming) process.\\

\noindent\emph{At the outline stage, do not write an abstract.
Writing the abstract is typically the last step before you submit a paper.}
}

% for single column abstract
\let\oldmaketitle\maketitle
\let\maketitle\relax
% }}}

% ----- document text -----
\begin{document}

% Format the abstract (make one column, much better looking...)
% {{{
\twocolumn[
\begin{@twocolumnfalse}
  \oldmaketitle
  \begin{abstract}
    \abstracttext
  \end{abstract}
\end{@twocolumnfalse}
]
% }}}

\section{Introduction}
% {{{
Molten salt electrorefining is an area of siginficant interest in the production and purification of many of the worlds important metals including aluminum, uranium, titanium and many others. 
Electrorefining in general is fairly well understood in aqueous systems however there is much more uncertainty in molten salt systems due to the difficulty of accurate measurments and testing. 
One means to overcome this obstacle is to develop simulations of the process and systems involved.

These simulations come in several different flavors and varieties.
The simplest form are pure thermodynamic models which focus on understanding the equilibrium behaivour of the metal in the elctrolyte. 
Other simulations utilize full 3D geometric information to capture the full breadth of mass transfer information.
Still other models strike a middle ground by simulating mass transfer in one dimension only. 

The ultimate gaol of these models is to obtain information that can be used to improve and understand the experiemntal and applied processes currently in use in many different industries. 
To obtain such information a model needs several different features and abilities. It needs to be fast enough to return information of a shorter time scale than the operation of an electrorefining cell.
It needs to catpure the most inportant aspects of physics and chemistry that could affect the electrorefining process.
Finally it needs to be experimentally validated so we know that the model adheres to reality.

A 1D model fits the criteria nicely. It is fast, while still capturing enough information about mass transfer and geometry to compare to experiments. 
Current 1D electrorefining models trace their origin to the TRAIL model with the most recent model coming from Cumberland in 2014. 
This model, dubbed ERAD, is focused on the potential development of passivation layers in Uranium electrorefining but is lacking key areas.
Most notably the model uses simple guesses as to the thickness of the diffusion layers which according to the authors of TRAIL is the "most important parameter of the model". 
ERAD was developed as a general ER model and thus has limited ability to be expermentally validated. 
Our efforts seek to develop an ER model that has closer ties to a specific system, allowing for direct experimental validation without a total sacrifice of generality.

The simulation developed in the paper calculates the mass transfer in the difusion layer, the diffusion layer thicknesses based on transport boundary layer theory and the anode activity as a funciton of the analyte mass fraction in the anode. 
Additionally the mass transport model incorporates transient migration and diffusion of every species in the system (molten salt components and analyte).

% }}}

\section{Methods}
% {{{
\emph{Don't write a methods section at the outline stage.}
Usually a methods section is pretty easy to write when you get to the ``full paper'' stage.
You typically write down the details of what you did (as succinctly as possible) so that someone else can replicate your work.
The hard part is balancing details with brevity.
I will help you with this when the time comes.

% }}}

\section{Results and Discussion}
% {{{

\subsection{Absolute boundary layer thickness}
\subsection{Relative boundary layer thickness}
\subsection{Boundary layer correlations relation to measurable properties}

\subsection{How to add figures}

I have included an example figure, i.e.\ Fig~\ref{fig-placeholder} to show you how to do this in LaTeX, and how to do it in your outline.
You can reference a figure by its name in LaTeX; this makes it easy when you add or remove a figure.

\begin{figure}[tbp]
  \includegraphics[width=3.25in]{fig-placeholder}
  \caption{This is a placeholder for the figure you should add. The caption should describe the figure and the necessary details.
It should not contain discussion of what the figure means.}
  \label{fig-placeholder}
\end{figure}

Your outline should include a list of points to make about your figure, e.g.\
\begin{itemize}
  \item How to add figures in LaTeX
  \item How to reference figures
  \item Adding bullet point lists in outlines
  \item Appropriate captions.
  \item Figure sizes
\end{itemize}

Note that most figures need to fit within a single column of a double column document.
The correct size in most journals is $3.25$ inches wide.
I like to make mine with an aspect ratio of $4\times3$, which gives a height of $2.44$ in, but you can choose an aspect ratio that you like.
If you need more space, you can make a double-wide figure of $6.5$ inches that spans two columns like Fig.~\ref{fig-doublewide}.
Keep the final figure size in mind while making your figures; it will save you a lot of time and effort in the end.
There are some example scripts in the \texttt{figure} directory that are hopefully useful to you.

\begin{figure*}[tbp]
  \includegraphics[width=6.5in]{fig-doublewide}
  \caption{This is a doublewide figure (spans both columns).}
  \label{fig-doublewide}
\end{figure*}

\subsection{How to add equations and tables}

You can add tables and equations to your paper as well as figures.
In the outline stage, you will probably have relatively few of these, but it is useful to see how to do this anyway.

Equations should be numbered, and should be inserted as part of a full sentence.
For example, the Pythagorean theorem
\begin{equation}
a^{2} = b^{2} + c^{2}
\end{equation}
could be an equation that you could include if you were going to write a paper on Fermat's last theorem.

Tables are inserted similarly to figures, as is shown by the example Table~\ref{tab-example}.
For more details, refer to the many online tutorials and examples of LaTeX tables.

\begin{table}
\caption{This is an example table}
\label{tab-example}
\begin{tabular}{cc}
\hline
Header 1 & Header 2 \\
\hline
R1, C1 & R1, C2 \\
R2, C1 & R2, C2 \\
\hline
\end{tabular}
\end{table}

% }}}

\section{Conclusion}
% {{{
Write a bulleted list here of the main conclusions of the paper.
These should include what the main, new contribution was to the problem outlined in the introduction.
The conclusion should be organized like an inverted funnel: start with the details of what you did and zoom out to state what the new state of the art is after you have completed your work.
The conclusion often terminates by stating what good ideas you have for future work.
% }}}

\begin{acknowledgement}
% {{{
% Acknowledgements of funding go here
We would like to acknowledge financial support from Brigham Young University and computational resources from the BYU Office of Research Computing.
% }}}
\end{acknowledgement}

\clearpage
\bibliography{refs}

\end{document}
