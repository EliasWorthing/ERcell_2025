
% ----- preamble -----
% {{{
\documentclass[journal=mamobx, layout=twocolumn]{achemso}

\usepackage{lipsum} % for generating dummy text
\usepackage[dvipsnames]{xcolor} % for colors
\usepackage{graphicx} % for figures
\usepackage{amsmath, amssymb} % for math
\usepackage{bm} % for bold math fonts
\usepackage{float} % Allows you to force figures to a certain location
\usepackage{stfloats} % Allows you to force figures to a certain location
\usepackage{xr} % crossreferencing main manuscript to and from supporting info
\externaldocument[SI:]{suppinfo}
% }}}

% --- Title, Author Info, Abstract Text ---
% {{{
\title{On the Development of a Molten Salt Electrorefining Simulation}
\author{Dakota S. Banks}
\affiliation{Chemical Engineering Department, Brigham Young University, Provo, Utah}
\date{\today}
\email{dakotasb@byu.edu}

\newcommand*{\abstracttext}{
% Put abstract text here
This document provides instructions for making an \emph{outline} of a paper.
The instructions here correspond to Step $3$ of the process described in the README and in Whitesides article.
Once you have an outline (and data of course), writing a paper is a relatively straightforward (but often time-consuming) process.\\

\noindent\emph{At the outline stage, do not write an abstract.
Writing the abstract is typically the last step before you submit a paper.}
}

% for single column abstract
\let\oldmaketitle\maketitle
\let\maketitle\relax
% }}}

% ----- document text -----
\begin{document}

% Format the abstract (make one column, much better looking...)
% {{{
\twocolumn[
\begin{@twocolumnfalse}
  \oldmaketitle
  \begin{abstract}
    \abstracttext
  \end{abstract}
\end{@twocolumnfalse}
]
% }}}

\section{Introduction}
% {{{
Molten salt electrorefining is an area of siginficant interest in the production and purification of many of the worlds important metals including aluminum, uranium, titanium and many others. 
Electrorefining in general is fairly well understood in aqueous systems however there is much more uncertainty in molten salt systems due to the difficulty of accurate measurments and testing. 
One means to overcome this obstacle is to develop simulations of the process and systems involved.

These simulations come in several different flavors and varieties.
The simplest form are pure thermodynamic models which focus on understanding the equilibrium behaivour of the metal in the elctrolyte. 
Other simulations utilize full 3D geometric information to capture the full breadth of mass transfer information.
Still other models strike a middle ground by simulating mass transfer in one dimension only. 

The ultimate gaol of these models is to obtain information that can be used to improve and understand the experiemntal and applied processes currently in use in many different industries. 
To obtain such information a model needs several different features and abilities. It needs to be fast enough to return information of a shorter time scale than the operation of an electrorefining cell.
It needs to catpure the most inportant aspects of physics and chemistry that could affect the electrorefining process.
Finally it needs to be experimentally validated so we know that the model adheres to reality.

A 1D model fits the criteria nicely. It is fast, while still capturing enough information about mass transfer and geometry to compare to experiments. 
Current 1D electrorefining models trace their origin to the TRAIL model with the most recent model coming from Cumberland in 2014. 
This model, dubbed ERAD, is focused on the potential development of passivation layers in Uranium electrorefining but is lacking key areas.
Most notably the model uses simple guesses as to the thickness of the diffusion layers which according to the authors of TRAIL is the "most important parameter of the model". 
ERAD was developed as a general ER model and thus has limited ability to be expermentally validated. 
Our efforts seek to develop an ER model that has closer ties to a specific system, allowing for direct experimental validation without a total sacrifice of generality.

The simulation developed in the paper calculates the mass transfer in the difusion layer, the diffusion layer thicknesses based on transport boundary layer theory and the anode activity as a funciton of the analyte mass fraction in the anode. 
Additionally the mass transport model incorporates transient migration and diffusion of every species in the system (molten salt components and analyte).

% }}}

\section{Methods}
% {{{
\subsection{Kinetics}
The kinetics of the surface reactions at each electrode and handled using a variation of the Bulter-Volmer equation for metal dissolution which is given by:
\begin{equation}
  i = i_0\left [\frac{X_s}{X_b}\exp\left (\frac{-\alpha nF\eta}{RT}\right ) 
- \frac{X_{m,s}}{X_{m,b}}\exp\left (\frac{(1-\alpha)nF\eta}{RT}\right )\right ]
\end{equation}
where $i_0 = nFk(X_b)^{(1-\alpha)}(X_{m,b})^\alpha$, $X$ represents a mass fraction with the subscript s for surface and b for bulk.
The m subscript is for the mass fraction in the metal on the electrode. $\eta$ represents the surface overpotential, $\alpha$ is the charge transfer coefficient which we have fixed at 0.5 for the time being.  
\subsection{Mass transfer}
In an ER cell there are three mass transfer vehicles; diffusion, migration, and convection. 
This model is concerned with ER cells that are continuously stirred. 
This causes two regions to form, a well-mixed inner region that comprises the majority of the cell and a pair of narrow regions near the surface of each electrode where convection is negligable.
The dffusion layers can be modeled with the following set of equations:

\begin{align}
    & \frac{dC_i}{dt} = z_iu_iF\nabla (C_i\nabla \phi) + D_i \nabla^2C_i, \label{eq:migration_diffusion}\\ 
    & \sum_i z_i C_i = 0 \label{eq:electroneutrality}
\end{align}
where the subscript $i$ refers to components ${1...n}$, $z$ is the charge number of each component and $\phi$ is the local electropotential (which is different than the applied potential).
These equations are all coupled through $\phi$ which is implicitly governed by Equation \ref{eq:electroneutrality}, the electroneutrality condition.
These equations are written with the assumption that the mobility, $u$, and the diffusion coefficients are constant with respect to space and is only strictly correct in diffuse systems. 
In a future paper this system will be revisited assuming a concentrated system but for this version of the model these equations suffice.

\subsection{Diffusion layer thickness}
To solve the mass transfer equations in the previous section, one must know the thickness of the diffusion layers.
This value is one of the most important parameters in the model since it directly controls the rate of diffusion. 
Other authors have estimated this value based on a range of assumptions and guesses but to validate the model experimentally it is important to tie this value to physically measurable parameters.

In the broader transport discipline much work has been done on the boundary layers that form in flowing fluids. 
Correlations have been developed for many different geormetries and flow parameters.
For this model two correlations have been used based on a commonly used geometry for critical material processing. 

The first correlation desribes the boundary layer formed on the surface of a disk with a fluid in rigid body rotation above it\cite{Smith_1972}.
\begin{equation}
    \delta = \left [\ln\left(\frac{b}{r}\right )\right]^\frac{1}{3}\frac{b}{\left(-\frac{F0}{9}\right)^\frac{1}{3}Re^\frac{1}{2}Sc^\frac{1}{3}}
    \label{eq:disk}
\end{equation}
In this correlation $Re = vb/\nu$ and $Sc = \nu/D$. $\nu$ is the kinematic viscosity, $v$ is the velocity, $b$ is the radius of the disc, $r$ is the position on that radius, $F0$ is a model parameter, $D$ is the diffusion coefficient, and $\delta$ is the diffusion layer thickness.
This equation is then averaged over the size of the disk to get a value for the diffusion layer thickness at the surface of the disk. 
This correlation represents the anode.

The second correlation is for rigid body rotation of a fluid inside of a hollow cylinder\cite{Rott1966}.
\begin{equation}
    \delta = \left ( \frac{\nu z}{vR^2}\right )^\frac{2}{5} R
    \label{eq:cylinder}
\end{equation}
In this equation $R$ is the cylinder radius, and $z$ is the vertical position in the cylinder. 
Again, this equation is averaged over the length of the cylinder to find a single value for the 1D mass transfer model. 
This correlation represents the cathode. 

Despite their simplicity, these correlations serve as a good approximation of the diffusion layers formed in this geometry. 
Additionally, these correlations can be replaced with others for a different geommetry, making this method capable of simulating a range of different cell designs. 

\subsection{Anode activity model} \label{sec:activity}
Since the anonde is molten metal alloy with a changing composistion over time it makes sense that the activity of the analyte in the anode will also change over time.
The model we are using to represent this change in activity is the Miedema model\cite{Miedema1980}.
\begin{multline}
        \ln\gamma_i = \frac{1}{RT}a_{ij}\Delta H_{ij}\left[1 + (1-x)\left[\frac{\left(1 - 2x\right)}{x\left(1-x\right)} -
\frac{\left[u_{i} \left({\phi}_{i} - {\phi}_{j}\right)\right]}{\left(1 + \left(1 - x\right)u_{i} \left({\phi}_{i} - {\phi}_{j}\right)\right)} + \right. \right.\\
\left. \left. \frac{\left[u_{j} \left({\phi}_{j} - {\phi}_{i}\right)\right]}{\left(1 + xu_{j} \left({\phi}_{j} - {\phi}_{i}\right)\right)} - 
\frac{V_{i} \left[1 + \left(1 - 2x\right)u_{i} \left({\phi}_{i} - {\phi}_{j}\right)  \right]+ V_{j}\left[-1 + \left(1 - 2x\right)u_{j} \left({\phi}_{j} - {\phi}_{i}\right)\right]} 
 {V_{j} \left(1 - x\right) \left(1 + xu_{j} \left({\phi}_{j} - {\phi}_{i}\right)\right) + V_{i}x \left(1 + \left(1 - x\right)u_{i} \left({\phi}_{i} - {\phi}_{j}\right) \right)}\right]\right]
\end{multline}
where 
\begin{equation}
    \Delta H_{ij} = f_{ij}\frac{x\left(1-x\right)\left(1 + xu_{j} \left({\phi}_{j} - {\phi}_{i}\right)\right)\left(1 + \left(1 - x\right)u_{i} \left({\phi}_{i} - {\phi}_{j}\right)\right)}{V_{j} \left(1 - x\right) \left(1 + xu_{j} \left({\phi}_{j} - {\phi}_{i}\right)\right) + V_{i}x \left(1 + \left(1 - x\right)u_{i} \left({\phi}_{i} - {\phi}_{j}\right) \right)}
\end{equation}
and
\begin{equation}
f_{ij} = \frac{2pV_iV_j\left \{q/p\left[n^{1/3}_i - n^{1/3}_j\right]^2-\left(\phi_i-\phi_j\right)^2-b(r/p)\right\}}{n^{-1/3}_i + n^{-1/3}_j}.
\end{equation}
Additionally, $x$ is the mass fraction of one component, $V_i$ and $V_j$ are molar volumes, $u_i$, $u_j$, $n_i$, $n_j$, $\phi_i$, $\phi_j$, $q$, $p$, and $b$ are model parameters, and $a_{ij}$ is a parameter related to the entropy.
One major limitation of this model is that it is unable to handle phase changes so if the anode solidifies at any point this model will not capture that. 
\section{Numerical methods}

% }}}

\section{Results and Discussion}
% {{{
\subsection{CV results comparison}
\subsection{Absolute boundary layer thickness}
\subsection{Relative boundary layer thickness}
\subsection{Boundary layer correlations relation to measurable properties}


Note that most figures need to fit within a single column of a double column document.
The correct size in most journals is $3.25$ inches wide.
I like to make mine with an aspect ratio of $4\times3$, which gives a height of $2.44$ in, but you can choose an aspect ratio that you like.
If you need more space, you can make a double-wide figure of $6.5$ inches that spans two columns like Fig.~\ref{fig-doublewide}.
Keep the final figure size in mind while making your figures; it will save you a lot of time and effort in the end.
There are some example scripts in the \texttt{figure} directory that are hopefully useful to you.

\begin{figure*}[tbp]
  \includegraphics[width=6.5in]{fig-doublewide}
  \caption{This is a doublewide figure (spans both columns).}
  \label{fig-doublewide}
\end{figure*}

% }}}

\section{Conclusion}
% {{{
Write a bulleted list here of the main conclusions of the paper.
These should include what the main, new contribution was to the problem outlined in the introduction.
The conclusion should be organized like an inverted funnel: start with the details of what you did and zoom out to state what the new state of the art is after you have completed your work.
The conclusion often terminates by stating what good ideas you have for future work.
% }}}

\begin{acknowledgement}
% {{{
% Acknowledgements of funding go here
We would like to acknowledge financial support from Brigham Young University and computational resources from the BYU Office of Research Computing.
% }}}
\end{acknowledgement}

\clearpage
\bibliography{refs}

\end{document}
